\documentclass[10pt,conference,letterpaper]{IEEEtran}

\usepackage[utf8]{inputenc}                   % Para escribir tildes y eñes
\usepackage[spanish]{babel}                   % Para que los títulos de figuras, tablas y otros estén en español
\usepackage{amsmath}      % Los paquetes ams son desarrollados por la American Mathematical Society
\usepackage{amsfonts}     % y mejoran la escritura de fórmulas y símbolos matemáticos.
\usepackage{amssymb}
\usepackage[dvips]{graphicx} % LaTeX
\DeclareGraphicsExtensions{.bmp,.png,.pdf,.jpg}
\usepackage{graphicx}     % Para insertar gráficas

\usepackage[justification=centering]{caption}
 \usepackage{epsfig}
 \usepackage{cite}
 \usepackage{amsmath,amsfonts,amssymb,standalone}
 \usepackage{color}
 \usepackage{tikz,forest}
 \usepackage{verbatim}
 \usepackage{xcolor}
 \usepackage{minted}
 \usepackage{epstopdf}
 %%%AGREGADOS
 \usepackage{color}
\definecolor{gray97}{gray}{.97}
\definecolor{gray75}{gray}{.75}
\definecolor{gray45}{gray}{.45}
\usepackage{adjustbox}
\usepackage{color}
\usepackage{graphicx}
\usepackage{epsfig}
\usepackage{multirow}
\usepackage{colortbl}
\definecolor{bl}{rgb}{0.6602, 0.796875, 0.8862}
\definecolor{cl}{rgb}{0.8359, 0.9140625, 0.96875}
\definecolor{bg}{rgb}{0.95,0.95,0.95}

\usepackage{unitsdef}	  % Para la presentación correcta de unidades
\renewcommand{\unitvaluesep}{\hspace*{2pt}}	% Redimensionamiento del espacio entre magnitud y unidad
\usepackage[colorlinks=true,urlcolor=blue,linkcolor=black,citecolor=blue]{hyperref}     % Para insertar hipervínculos y marcadores
\usepackage{float}		% Para ubicar las tablas y figuras justo después del texto
\usepackage{booktabs}	% Para hacer tablas más estilizadas
\batchmode
\pagenumbering{arabic}
\usepackage{lastpage}
\usepackage{svg}
%\userpackage{pdfpages} 
\usepackage{longtable}
\usepackage{array}
\usepackage{pdfpages}
\newtheorem{theorem}{Theorem}
\setcounter{page}{1}

%%%ENCABEZADOS
\usepackage{fancyhdr}	% Para manejar los encabezados y pies de página
\pagestyle{fancy}		% Contenido de los encabezados y pies de pagina
\lhead{MP-6104 Procesamiento Digital de Señales}
%\chead{Julio/21/2020}
\rhead{III Cuatrimestre 2020}
\lfoot{Escuela de Ingeniería Electrónica}
\cfoot{\thepage\ de \pageref{LastPage}}
\rfoot{Instituto Tecnológico de Costa Rica}
\author{
\IEEEauthorblockN{\textbf{$\cdot$  David Martínez García}}
\IEEEauthorblockA{2013005337\\
\texttt{david.martinez@estudiantec.cr}}
\and
\IEEEauthorblockN{\textbf{$\cdot$  José Martínez Hernández}}
\IEEEauthorblockA{2020426476\\
\texttt{jpmh.1309@estudiantec.cr}}}
\title{
		\textsc{\LARGE Instituto Tecnológico de Costa Rica}\\[-0.4 cm]	
		\textsc{\Large Escuela de Ingeniería Electrónica}\\[-0.4 cm]
		\textsc{\large MP-6104 Procesamiento Digital de Señales}\\[-0.5 cm]
		\rule{\linewidth}{0.2 mm} \\[-0.1 cm]
		\textsc{\LARGE \bfseries Proyecto #1: Esteganografía por Enmascaramiento con Eco \\[-0.7 cm]} 
		\rule{\linewidth}{0.2 mm} \\[-1.0 cm]
}

\begin{document}

%%%%%%%%%%%%%%%%%%%%%%%%%%%%%%%%%%%%%%%%%%%%%%%%%%%%%%%%%%%%%%%%
%%%PORTADA
\pdfbookmark[1]{Portada}{portada} 	% Marcador para el título
\maketitle							% Título
%%%%%%%%%%%%%%%%%%%%%%%%%%%%%%%%%%%%%%%%%%%%%%%%%%%%%%%%%%%%%%%%

%%% RESUMEN
%%% RESUMEN
%%%%%%%%%%%%%%%%%%%%%%%%%%%%%%%%%%%%%%%%%%%%%%
\begin{abstract}
En este proyecto se realizo el particionamiento HW/SW del modelo de un drone básico comercial y además se construyo los modelos de abstracción  PV, LT, AT para modelar cada uno de los submódulos que componen un drone. Además se realizo un banco de pruebas para validar los resultados obtenidos. 
\end{abstract}


%%%%%%%%%%%%%%%%%%%%%%%%%%%%%%%%%%%%%%%%%%%%%%%%%%%%%%%%
%%% PALABRAS CLAVES
\vspace*{0.0cm}
\begin{small}
\textbf{Palabras claves}: \textit{Drone, SystemC, HW, SW, Power Managmente Unit (PMU), Flight Control Unit (FCU), Remote Control (RC), Electronic Speed Cnntrol (ESC).}
\end{small}


%%% INTRODUCCIÓN
%%% INTRODUCCIÓN
%%%%%%%%%%%%%%%%%%%%%%%%%%%%%%%%%%%%%%%%%%%%%%
\section{\textbf{Introducción}}
%%%%%%%%%%%%%%%%%%%%%%%%%%%%%%%%%%%%%%%%%%%%%%



%%% PROYECTO 1
%\input{zd_Proyecto1.tex}

%%% PROYECTO 2
%\input{zd_Proyecto2.tex}

%%% PROYECTO 3
%%%%%%%%%%%%%%%%%%%%%%%%%%%%%%%%%%%%%%%%%%%%%%%%%%%
%%% SYSTEM C AMS 
%%%%%%%%%%%%%%%%%%%%%%%%%%%%%%%%%%%%%%%%%%%%%%%%%%
\section{\textbf{System C AMS}}

El MMIO se conecta a los sistemas analógicos. Entre los sistemas analogicos diseñados se encuentran los DAC, los motores DC y emisor/receptor del control remoto.

El motor DC se modelo de la siguiente manera:

\begin{equation}
    H(s) = \frac{1}{1+0.003s}
    
Y para el emisor/receptor del control remoto se usa una señal PPM modulada con BASK, en las hsiguientes figuras se puede observar los diagramas del sistemas desarrollado.
 
\end{equation}
\begin{figure}[H]
\centering
\includegraphics[width=0.8\linewidth]{img/RC_TX.png}
\caption{System C AMS RC TX}
\label{F:rc_tx}
\end{figure}

\begin{figure}[H]
\centering
\includegraphics[width=0.8\linewidth]{img/RC_RX.png}
\caption{System C AMS RC RX}
\label{F:rc_rx}
\end{figure}

%%% PROYECTO 4
%%%%%%%%%%%%%%%%%%%%%%%%%%%%%%%%%%%%%%%%%%%%%%%%%%%
%%% GEM 5 
%%%%%%%%%%%%%%%%%%%%%%%%%%%%%%%%%%%%%%%%%%%%%%%%%%
\section{\textbf{Gem 5}}

%%% PROYECTO 5
%\input{zd_Proyecto5.tex}

%%% REFERENCIAS
\begin{thebibliography}{1}
\bibitem{R1} Basile, A.  (2020). \emph{\textbf{Getting Started with
ST Drone Kit. }} ST Electronics.
\end{thebibliography}

%%% APENDICES
\appendix 
%%%%%%%%%%%%%%%%%%%%%%%%%%%%%%%%%%%%%%%%%%%%%%
\subsection{Código desarrollado} \label{A:Code}
%%%%%%%%%%%%%%%%%%%%%%%%%%%%%%%%%%%%%%%%%%%%%%

\href{https://github.com/coursesmichaelgruner/dsp-2020-3-proyecto1-david_jose}{dsp-2020-3-proyecto1-david\_jose}
%%%%%%%%%%%%%%%%%%%%%%%%%%%%%%%%%%%%%%%%%%%%%%
\end{document}